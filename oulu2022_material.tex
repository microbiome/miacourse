% Options for packages loaded elsewhere
\PassOptionsToPackage{unicode}{hyperref}
\PassOptionsToPackage{hyphens}{url}
%
\documentclass[
  oneside]{book}
\usepackage{amsmath,amssymb}
\usepackage{lmodern}
\usepackage{iftex}
\ifPDFTeX
  \usepackage[T1]{fontenc}
  \usepackage[utf8]{inputenc}
  \usepackage{textcomp} % provide euro and other symbols
\else % if luatex or xetex
  \usepackage{unicode-math}
  \defaultfontfeatures{Scale=MatchLowercase}
  \defaultfontfeatures[\rmfamily]{Ligatures=TeX,Scale=1}
\fi
% Use upquote if available, for straight quotes in verbatim environments
\IfFileExists{upquote.sty}{\usepackage{upquote}}{}
\IfFileExists{microtype.sty}{% use microtype if available
  \usepackage[]{microtype}
  \UseMicrotypeSet[protrusion]{basicmath} % disable protrusion for tt fonts
}{}
\makeatletter
\@ifundefined{KOMAClassName}{% if non-KOMA class
  \IfFileExists{parskip.sty}{%
    \usepackage{parskip}
  }{% else
    \setlength{\parindent}{0pt}
    \setlength{\parskip}{6pt plus 2pt minus 1pt}}
}{% if KOMA class
  \KOMAoptions{parskip=half}}
\makeatother
\usepackage{xcolor}
\usepackage[top=30mm,left=15mm]{geometry}
\usepackage{color}
\usepackage{fancyvrb}
\newcommand{\VerbBar}{|}
\newcommand{\VERB}{\Verb[commandchars=\\\{\}]}
\DefineVerbatimEnvironment{Highlighting}{Verbatim}{commandchars=\\\{\}}
% Add ',fontsize=\small' for more characters per line
\usepackage{framed}
\definecolor{shadecolor}{RGB}{248,248,248}
\newenvironment{Shaded}{\begin{snugshade}}{\end{snugshade}}
\newcommand{\AlertTok}[1]{\textcolor[rgb]{0.94,0.16,0.16}{#1}}
\newcommand{\AnnotationTok}[1]{\textcolor[rgb]{0.56,0.35,0.01}{\textbf{\textit{#1}}}}
\newcommand{\AttributeTok}[1]{\textcolor[rgb]{0.77,0.63,0.00}{#1}}
\newcommand{\BaseNTok}[1]{\textcolor[rgb]{0.00,0.00,0.81}{#1}}
\newcommand{\BuiltInTok}[1]{#1}
\newcommand{\CharTok}[1]{\textcolor[rgb]{0.31,0.60,0.02}{#1}}
\newcommand{\CommentTok}[1]{\textcolor[rgb]{0.56,0.35,0.01}{\textit{#1}}}
\newcommand{\CommentVarTok}[1]{\textcolor[rgb]{0.56,0.35,0.01}{\textbf{\textit{#1}}}}
\newcommand{\ConstantTok}[1]{\textcolor[rgb]{0.00,0.00,0.00}{#1}}
\newcommand{\ControlFlowTok}[1]{\textcolor[rgb]{0.13,0.29,0.53}{\textbf{#1}}}
\newcommand{\DataTypeTok}[1]{\textcolor[rgb]{0.13,0.29,0.53}{#1}}
\newcommand{\DecValTok}[1]{\textcolor[rgb]{0.00,0.00,0.81}{#1}}
\newcommand{\DocumentationTok}[1]{\textcolor[rgb]{0.56,0.35,0.01}{\textbf{\textit{#1}}}}
\newcommand{\ErrorTok}[1]{\textcolor[rgb]{0.64,0.00,0.00}{\textbf{#1}}}
\newcommand{\ExtensionTok}[1]{#1}
\newcommand{\FloatTok}[1]{\textcolor[rgb]{0.00,0.00,0.81}{#1}}
\newcommand{\FunctionTok}[1]{\textcolor[rgb]{0.00,0.00,0.00}{#1}}
\newcommand{\ImportTok}[1]{#1}
\newcommand{\InformationTok}[1]{\textcolor[rgb]{0.56,0.35,0.01}{\textbf{\textit{#1}}}}
\newcommand{\KeywordTok}[1]{\textcolor[rgb]{0.13,0.29,0.53}{\textbf{#1}}}
\newcommand{\NormalTok}[1]{#1}
\newcommand{\OperatorTok}[1]{\textcolor[rgb]{0.81,0.36,0.00}{\textbf{#1}}}
\newcommand{\OtherTok}[1]{\textcolor[rgb]{0.56,0.35,0.01}{#1}}
\newcommand{\PreprocessorTok}[1]{\textcolor[rgb]{0.56,0.35,0.01}{\textit{#1}}}
\newcommand{\RegionMarkerTok}[1]{#1}
\newcommand{\SpecialCharTok}[1]{\textcolor[rgb]{0.00,0.00,0.00}{#1}}
\newcommand{\SpecialStringTok}[1]{\textcolor[rgb]{0.31,0.60,0.02}{#1}}
\newcommand{\StringTok}[1]{\textcolor[rgb]{0.31,0.60,0.02}{#1}}
\newcommand{\VariableTok}[1]{\textcolor[rgb]{0.00,0.00,0.00}{#1}}
\newcommand{\VerbatimStringTok}[1]{\textcolor[rgb]{0.31,0.60,0.02}{#1}}
\newcommand{\WarningTok}[1]{\textcolor[rgb]{0.56,0.35,0.01}{\textbf{\textit{#1}}}}
\usepackage{longtable,booktabs,array}
\usepackage{calc} % for calculating minipage widths
% Correct order of tables after \paragraph or \subparagraph
\usepackage{etoolbox}
\makeatletter
\patchcmd\longtable{\par}{\if@noskipsec\mbox{}\fi\par}{}{}
\makeatother
% Allow footnotes in longtable head/foot
\IfFileExists{footnotehyper.sty}{\usepackage{footnotehyper}}{\usepackage{footnote}}
\makesavenoteenv{longtable}
\usepackage{graphicx}
\makeatletter
\def\maxwidth{\ifdim\Gin@nat@width>\linewidth\linewidth\else\Gin@nat@width\fi}
\def\maxheight{\ifdim\Gin@nat@height>\textheight\textheight\else\Gin@nat@height\fi}
\makeatother
% Scale images if necessary, so that they will not overflow the page
% margins by default, and it is still possible to overwrite the defaults
% using explicit options in \includegraphics[width, height, ...]{}
\setkeys{Gin}{width=\maxwidth,height=\maxheight,keepaspectratio}
% Set default figure placement to htbp
\makeatletter
\def\fps@figure{htbp}
\makeatother
\setlength{\emergencystretch}{3em} % prevent overfull lines
\providecommand{\tightlist}{%
  \setlength{\itemsep}{0pt}\setlength{\parskip}{0pt}}
\setcounter{secnumdepth}{5}
\usepackage{booktabs}
\ifLuaTeX
  \usepackage{selnolig}  % disable illegal ligatures
\fi
\usepackage[]{natbib}
\bibliographystyle{apalike}
\IfFileExists{bookmark.sty}{\usepackage{bookmark}}{\usepackage{hyperref}}
\IfFileExists{xurl.sty}{\usepackage{xurl}}{} % add URL line breaks if available
\urlstyle{same} % disable monospaced font for URLs
\hypersetup{
  pdftitle={Microbiome data science with R/Bioconductor},
  hidelinks,
  pdfcreator={LaTeX via pandoc}}

\title{Microbiome data science with R/Bioconductor}
\author{}
\date{\vspace{-2.5em}}

\begin{document}
\maketitle

{
\setcounter{tocdepth}{1}
\tableofcontents
}
\hypertarget{overview}{%
\chapter{Overview}\label{overview}}

\hypertarget{schedule}{%
\section{Schedule}\label{schedule}}

Download the \href{SPARCworkshop2023schedule.pdf}{full schedule}.

The schedule is summarized as follows.

\begin{itemize}
\tightlist
\item
  Day 1 (Tue) - Symposium; online lectures and no hands-on session
\item
  Day 2 (Wed) - Online lectures; hands-on session on \textbf{R/Bioconductor framework}
\item
  Day 3 (Thu) - Online lectures; hands-on session on \textbf{microbiome data analysis methods}
\item
  Day 4 (Fri) - Online lectures; advanced microbiome \textbf{data analysis methods}
\end{itemize}

\hypertarget{learning-goals}{%
\section{Learning goals}\label{learning-goals}}

This course will teach the \textbf{basics of microbiome data analysis and
integration with R/Bioconductor}, a popular open source environment
for scientific data analysis.

You will get an overview of the reproducible data analysis workflow,
with recent examples from published studies.

After the course you will know how to approach new tasks in microbiome
data science by utilizing the available R tools and documentation. In
particular, you understand the concepts of data containers,
reproducible workflows, and standard concepts in microbiome data
analysis.

\hypertarget{target-audience}{%
\section{Target audience}\label{target-audience}}

The course is primarily designed for advanced MSc and PhD students,
Postdocs, and biomedical researchers who wish to learn and develop new
skills in scientific programming and microbiome data science.
Academic students and researchers from Finland and abroad are welcome
and encouraged to apply. The course has limited capacity, and priority
will given for local students.

\textbf{Expected background} Earlier experience with R or another
programming language is expected. The teaching format allows
adaptations according to the student's learning speed.

\hypertarget{learning-material}{%
\section{Learning material}\label{learning-material}}

The teaching builds on the open online tutorial, Orchestrating
Microbiome Analysis (\url{https://microbiome.github.io/OMA}). The openly
licensed teaching material will be available online during and after
the course, following \href{https://edition.fi/tsv/catalog/book/421}{recommendations on open education}.

The training material walks you through the standard steps of
microbiome data analysis covering data import, processing,
exploration, analysis, visualization, reproducible reporting, and best
practices in open science. We teach generic data analytical skills
that are applicable to common data analysis tasks encountered in
modern omics research. The teaching format allows adaptations
according to the student's learning speed.

Link to online Gitter chat:
\url{https://microbiome.github.io}

\hypertarget{checklist-preparing-for-the-course}{%
\chapter{Checklist: preparing for the course}\label{checklist-preparing-for-the-course}}

\hypertarget{questionnaire-on-the-background-of-participants}{%
\section{Questionnaire on the background of participants}\label{questionnaire-on-the-background-of-participants}}

Fill in the anonymous \href{https://forms.gle/XZdiEyGyYtLKYwqp8}{questionnaire}.

This information will help us to understand the background of the
participants better, and adjust teaching accordingly.

\hypertarget{packages}{%
\section{Installing the required R/Bioconductor packages}\label{packages}}

Install the required software in advance.

\begin{itemize}
\item
  \href{https://www.r-project.org/}{R (it is critical to use the latest official release!)}
\item
  \href{https://www.rstudio.com/products/rstudio/download/}{RStudio};
  choose ``Rstudio Desktop'' to download the latest version. For further
  details, check the \href{https://www.rstudio.com/}{Rstudio home page}.
\item
  Install and load the required R packages (see below)
\item
  After a successful installation you can start with the
  case study examples in the training material
\end{itemize}

\hypertarget{required-rbioconductor-packages}{%
\subsection{Required R/Bioconductor packages}\label{required-rbioconductor-packages}}

This section shows how to install and load all required packages into
the R session, if needed. Only uninstalled packages are installed.

\begin{Shaded}
\begin{Highlighting}[]
\CommentTok{\# List of packages that we need from cran and bioc }
\NormalTok{cran\_pkg }\OtherTok{\textless{}{-}} \FunctionTok{c}\NormalTok{(}\StringTok{"BiocManager"}\NormalTok{, }\StringTok{"bookdown"}\NormalTok{, }\StringTok{"dplyr"}\NormalTok{, }\StringTok{"ecodist"}\NormalTok{, }\StringTok{"ggplot2"}\NormalTok{, }
              \StringTok{"gridExtra"}\NormalTok{, }\StringTok{"kableExtra"}\NormalTok{,  }\StringTok{"knitr"}\NormalTok{, }\StringTok{"scales"}\NormalTok{, }\StringTok{"vegan"}\NormalTok{, }\StringTok{"matrixStats"}\NormalTok{)}
\NormalTok{bioc\_pkg }\OtherTok{\textless{}{-}} \FunctionTok{c}\NormalTok{(}\StringTok{"yulab.utils"}\NormalTok{,}\StringTok{"ggtree"}\NormalTok{,}\StringTok{"ANCOMBC"}\NormalTok{, }\StringTok{"ape"}\NormalTok{, }\StringTok{"DESeq2"}\NormalTok{, }\StringTok{"DirichletMultinomial"}\NormalTok{, }\StringTok{"mia"}\NormalTok{, }\StringTok{"miaViz"}\NormalTok{, }\StringTok{"miaSim"}\NormalTok{)}
\NormalTok{github\_pkg }\OtherTok{\textless{}{-}} \FunctionTok{c}\NormalTok{(}\StringTok{"miaTime"}\NormalTok{)}


\CommentTok{\# Get those packages that are already installed}
\NormalTok{cran\_pkg\_already\_installed }\OtherTok{\textless{}{-}}\NormalTok{ cran\_pkg[ cran\_pkg }\SpecialCharTok{\%in\%} \FunctionTok{installed.packages}\NormalTok{() ]}
\NormalTok{bioc\_pkg\_already\_installed }\OtherTok{\textless{}{-}}\NormalTok{ bioc\_pkg[ bioc\_pkg }\SpecialCharTok{\%in\%} \FunctionTok{installed.packages}\NormalTok{() ]}
\NormalTok{github\_pkg\_already\_installed }\OtherTok{\textless{}{-}}\NormalTok{ github\_pkg[ github\_pkg }\SpecialCharTok{\%in\%} \FunctionTok{installed.packages}\NormalTok{() ]}

\CommentTok{\# Get those packages that need to be installed}
\NormalTok{cran\_pkg\_to\_be\_installed }\OtherTok{\textless{}{-}} \FunctionTok{setdiff}\NormalTok{(cran\_pkg, cran\_pkg\_already\_installed)}
\NormalTok{bioc\_pkg\_to\_be\_installed }\OtherTok{\textless{}{-}} \FunctionTok{setdiff}\NormalTok{(bioc\_pkg, bioc\_pkg\_already\_installed)}
\NormalTok{github\_pkg\_to\_be\_installed }\OtherTok{\textless{}{-}} \FunctionTok{setdiff}\NormalTok{(github\_pkg, github\_pkg\_already\_installed)}

\CommentTok{\# Reorders bioc packages, so that mia and miaViz are first}
\NormalTok{bioc\_pkg }\OtherTok{\textless{}{-}} \FunctionTok{c}\NormalTok{(bioc\_pkg[ bioc\_pkg }\SpecialCharTok{\%in\%} \FunctionTok{c}\NormalTok{(}\StringTok{"mia"}\NormalTok{, }\StringTok{"miaViz"}\NormalTok{) ], }
\NormalTok{              bioc\_pkg[ }\SpecialCharTok{!}\NormalTok{bioc\_pkg }\SpecialCharTok{\%in\%} \FunctionTok{c}\NormalTok{(}\StringTok{"mia"}\NormalTok{, }\StringTok{"miaViz"}\NormalTok{) ] ) }

\CommentTok{\# Combine to one vector}
\NormalTok{packages }\OtherTok{\textless{}{-}} \FunctionTok{c}\NormalTok{(bioc\_pkg, cran\_pkg)}
\NormalTok{packages\_to\_install }\OtherTok{\textless{}{-}} \FunctionTok{c}\NormalTok{( bioc\_pkg\_to\_be\_installed, cran\_pkg\_to\_be\_installed, cran\_pkg\_to\_be\_installed)}
\end{Highlighting}
\end{Shaded}

\begin{Shaded}
\begin{Highlighting}[]
\CommentTok{\# If there are packages that need to be installed, install them }
\ControlFlowTok{if}\NormalTok{( }\FunctionTok{length}\NormalTok{(packages\_to\_install) ) \{}
\NormalTok{   BiocManager}\SpecialCharTok{::}\FunctionTok{install}\NormalTok{(packages\_to\_install)}
\NormalTok{\}}
\end{Highlighting}
\end{Shaded}

Now all required packages are installed, so let's load them into the session.
Some function names occur in multiple packages. That is why miaverse's packages
mia and miaViz are prioritized. Packages that are loaded first have higher priority.

\begin{Shaded}
\begin{Highlighting}[]
\CommentTok{\# Loading all packages into session. Returns true if package was successfully loaded.}
\NormalTok{loaded }\OtherTok{\textless{}{-}} \FunctionTok{sapply}\NormalTok{(packages, require, }\AttributeTok{character.only =} \ConstantTok{TRUE}\NormalTok{)}
\FunctionTok{as.data.frame}\NormalTok{(loaded)}
\end{Highlighting}
\end{Shaded}

\hypertarget{reading-and-support}{%
\section{Reading and support}\label{reading-and-support}}

\begin{itemize}
\item
  View the short online videos on \href{https://www.youtube.com/playlist?list=PLjiXAZO27elAJEptP59BN3whVJ61XIkST}{R/Bioconductor microbiome data science tools}.
\item
  Check the Appendix chapter of the \href{https://microbiome.github.io/OMA}{OMA
  book}. In particular, read Chapter
  15.3 on reproducible reporting.
\item
  \textbf{You can run the workflows by simply copy-pasting the examples.} For
  further, advanced material, you can test and modify further examples
  from the book, and apply these techniques to your own data.
\item
  Online support on installation and other matters, join us at \href{https://gitter.im/microbiome/miaverse?utm_source=badge\&utm_medium=badge\&utm_campaign=pr-badge\&utm_content=badge}{Gitter}
\end{itemize}

\hypertarget{acknowledgments}{%
\chapter{Acknowledgments}\label{acknowledgments}}

\hypertarget{teachers-and-organizers}{%
\section{Teachers and organizers}\label{teachers-and-organizers}}

\begin{itemize}
\item
  \href{https://datascience.utu.fi}{Leo Lahti} is the main teacher and
  Associate Professor in Data Science at the University of Turku,
  Finland, with specialization on microbiome research.
\item
  Prof.~Richa Ashma; local organizer.
\item
  Doctoral candidate Renuka Potbhare; course assistant
\end{itemize}

\hypertarget{support}{%
\section{Support}\label{support}}

The course is funded by SPARC, and jointly organized by:

\begin{itemize}
\tightlist
\item
  Savitribai Phule Pune University, Pune, India
\item
  Department of Computing, University of Turku, Finland
\item
  CompLifeSci Biocity Research Program, Turku, Finland
\end{itemize}

The teaching materials have been developed with support from

\begin{itemize}
\tightlist
\item
  ML4microbiome COST action
\item
  Horizon/RIA project FindingPheno
\item
  CompLifeSci Biocity Research Program, Turku, Finland
\item
  Turku University Foundation
\item
  Academy of Finland
\end{itemize}

\textbf{Citation} We thank all \href{https://microbiome.github.io}{developers and contributors} who have contributed open resources that supported the development of the training material. Kindly cite the course material as \citet{miacourse}

\textbf{License and source code}

All material is released under the open \href{LICENSE}{CC BY-NC-SA 3.0
License} and available online during and after the course,
following the \href{https://avointiede.fi/fi/linjaukset-ja-aineistot/kotimaiset-linjaukset/oppimisen-ja-oppimateriaalien-avoimuuden-linjaus}{recommendations on open teaching
materials}
of the national open science coordination in Finland**.

  \bibliography{packages.bib}

\end{document}
